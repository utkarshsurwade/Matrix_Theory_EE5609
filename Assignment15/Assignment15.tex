
\documentclass[journal,12pt]{IEEEtran}
\usepackage{longtable}
\usepackage{setspace}
\usepackage{gensymb}
\singlespacing
\usepackage[cmex10]{amsmath}
\newcommand\myemptypage{
	\null
	\thispagestyle{empty}
	\addtocounter{page}{-1}
	\newpage
}
\usepackage{amsthm}
\usepackage{mdframed}
\usepackage{mathrsfs}
\usepackage{txfonts}
\usepackage{stfloats}
\usepackage{bm}
\usepackage{cite}
\usepackage{cases}
\usepackage{subfig}

\usepackage{longtable}
\usepackage{multirow}


\usepackage{enumitem}
\usepackage{mathtools}
\usepackage{steinmetz}
\usepackage{tikz}
\usepackage{circuitikz}
\usepackage{verbatim}
\usepackage{tfrupee}
\usepackage[breaklinks=true]{hyperref}
\usepackage{graphicx}
\usepackage{tkz-euclide}

\usetikzlibrary{calc,math}
\usepackage{listings}
    \usepackage{color}                                            %%
    \usepackage{array}                                            %%
    \usepackage{longtable}                                        %%
    \usepackage{calc}                                             %%
    \usepackage{multirow}                                         %%
    \usepackage{hhline}                                           %%
    \usepackage{ifthen}                                           %%
    \usepackage{lscape}     
\usepackage{multicol}
\usepackage{chngcntr}

\DeclareMathOperator*{\Res}{Res}

\renewcommand\thesection{\arabic{section}}
\renewcommand\thesubsection{\thesection.\arabic{subsection}}
\renewcommand\thesubsubsection{\thesubsection.\arabic{subsubsection}}

\renewcommand\thesectiondis{\arabic{section}}
\renewcommand\thesubsectiondis{\thesectiondis.\arabic{subsection}}
\renewcommand\thesubsubsectiondis{\thesubsectiondis.\arabic{subsubsection}}


\hyphenation{op-tical net-works semi-conduc-tor}
\def\inputGnumericTable{}                                 %%

\lstset{
%language=C,
frame=single, 
breaklines=true,
columns=fullflexible
}
\begin{document}
\onecolumn

\newtheorem{theorem}{Theorem}[section]
\newtheorem{problem}{Problem}
\newtheorem{proposition}{Proposition}[section]
\newtheorem{lemma}{Lemma}[section]
\newtheorem{corollary}[theorem]{Corollary}
\newtheorem{example}{Example}[section]
\newtheorem{definition}[problem]{Definition}

\newcommand{\BEQA}{\begin{eqnarray}}
\newcommand{\EEQA}{\end{eqnarray}}
\newcommand{\define}{\stackrel{\triangle}{=}}
\bibliographystyle{IEEEtran}
\raggedbottom
\setlength{\parindent}{0pt}
\providecommand{\mbf}{\mathbf}
\providecommand{\pr}[1]{\ensuremath{\Pr\left(#1\right)}}
\providecommand{\qfunc}[1]{\ensuremath{Q\left(#1\right)}}
\providecommand{\sbrak}[1]{\ensuremath{{}\left[#1\right]}}
\providecommand{\lsbrak}[1]{\ensuremath{{}\left[#1\right.}}
\providecommand{\rsbrak}[1]{\ensuremath{{}\left.#1\right]}}
\providecommand{\brak}[1]{\ensuremath{\left(#1\right)}}
\providecommand{\lbrak}[1]{\ensuremath{\left(#1\right.}}
\providecommand{\rbrak}[1]{\ensuremath{\left.#1\right)}}
\providecommand{\cbrak}[1]{\ensuremath{\left\{#1\right\}}}
\providecommand{\lcbrak}[1]{\ensuremath{\left\{#1\right.}}
\providecommand{\rcbrak}[1]{\ensuremath{\left.#1\right\}}}
\theoremstyle{remark}
\newtheorem{rem}{Remark}
\newcommand{\sgn}{\mathop{\mathrm{sgn}}}
\providecommand{\abs}[1]{\left\vert#1\right\vert}
\providecommand{\res}[1]{\Res\displaylimits_{#1}} 
\providecommand{\norm}[1]{\left\lVert#1\right\rVert}
%\providecommand{\norm}[1]{\lVert#1\rVert}
\providecommand{\mtx}[1]{\mathbf{#1}}
\providecommand{\mean}[1]{E\left[ #1 \right]}
\providecommand{\fourier}{\overset{\mathcal{F}}{ \rightleftharpoons}}
%\providecommand{\hilbert}{\overset{\mathcal{H}}{ \rightleftharpoons}}
\providecommand{\system}{\overset{\mathcal{H}}{ \longleftrightarrow}}
	%\newcommand{\solution}[2]{\textbf{Solution:}{#1}}
\newcommand{\solution}{\noindent \textbf{Solution: }}
\newcommand{\cosec}{\,\text{cosec}\,}
\providecommand{\dec}[2]{\ensuremath{\overset{#1}{\underset{#2}{\gtrless}}}}
\newcommand{\myvec}[1]{\ensuremath{\begin{pmatrix}#1\end{pmatrix}}}
\newcommand{\mydet}[1]{\ensuremath{\begin{vmatrix}#1\end{vmatrix}}}
\numberwithin{equation}{subsection}
\makeatletter
\@addtoreset{figure}{problem}
\makeatother
\let\StandardTheFigure\thefigure
\let\vec\mathbf
\renewcommand{\thefigure}{\theproblem}
\def\putbox#1#2#3{\makebox[0in][l]{\makebox[#1][l]{}\raisebox{\baselineskip}[0in][0in]{\raisebox{#2}[0in][0in]{#3}}}}
     \def\rightbox#1{\makebox[0in][r]{#1}}
     \def\centbox#1{\makebox[0in]{#1}}
     \def\topbox#1{\raisebox{-\baselineskip}[0in][0in]{#1}}
     \def\midbox#1{\raisebox{-0.5\baselineskip}[0in][0in]{#1}}
\vspace{3cm}
\title{Assignment 15}
\author{Utkarsh Surwade\\AI20MTECH11004}
\maketitle
\bigskip
\renewcommand{\thefigure}{\theenumi}
\renewcommand{\thetable}{\theenumi}
Download latex-tikz codes from 
%
\begin{lstlisting}
https://github.com/utkarshsurwade/Matrix_Theory_EE5609/tree/master/codes
\end{lstlisting}
%
 
\section{\textbf{Problem}}
Let $\vec{A}$ be a real matrix with characteristic polynomial $(x-1)^3$.Pick the correct statements from below:\\
1. $\vec{A}$ is necessarily diagonalizable.\\
2. If the minimal polynomial of $\vec{A}$ is $(x-1)^3$, then $\vec{A}$ is diagonalizale.\\
3. Characteristic polynomial of $\vec{A}^2$ is $(x-1)^3$.\\
4. If $\vec{A}$ has exactly two Jordan blocks, then $(\vec{A-I})^2$ is diagonalizable.

\section{\textbf{Definitions}}
\renewcommand{\thetable}{1}
\begin{table}[ht!]
\centering
\begin{tabular}{|c|l|}
    \hline
	\multirow{3}{*}{Characteristic Polynomial} 
	& \\
	& For an $n\times n$ matrix $\vec{A}$, characteristic polynomial is defined by,\\
	&\\
	& $\qquad\qquad\qquad p\brak{x}=\mydet{x\Vec{I}-\Vec{A}}$\\
	&\\
	\hline
	\multirow{3}{*}{Cayley-Hamilton Theorem}
    &\\
    & If $p\brak{x}$ is the characteristic polynomial of an $n\times n$ matrix $\vec{A}$, then,\\
    &\\
    &$\qquad \qquad \qquad p\brak{\vec{A}}=\vec{0}$\\
    &\\
    \hline
	\multirow{3}{*}{Minimal Polynomial} 
	&\\
	& Minimal polynomial $m\brak{x}$ is the smallest factor of characteristic polynomial\\
	& $p\brak{x}$ such that,\\
	&\\
	& $\qquad \qquad \qquad m\brak{\vec{A}}=0$\\
	& \\
	& Every root of characteristic polynomial should be the root of minimal\\
	& polynomial\\
	&\\
    \hline
\end{tabular}
\label{table:1}
    \caption{Definitions}
\end{table}
\newpage

\section{\textbf{Explanation}}
\renewcommand{\thetable}{2}
\begin{longtable}{|l|l|}
\hline
\multirow{3}{*}{} & \\
Statement&Solution\\
\hline
&\\
1.&\\
&\parbox{6cm}{\begin{align*}
    \mbox{Given that, }p(x)&=(x-1)^3\\
    \mbox{Let us consider }m(x)&=(x-1)^2\\
    \therefore \vec{J}&=\myvec{1&1&0\\0&1&0\\0&0&1}
\end{align*}}\\
&A matrix is diagonalizable iff its jordan form is a diagonal matrix.\\
&Since $\vec{J}$ is not diagonizable therefore $\vec{A}$ is not diagonizable.\\
&\\
\hline
&\\
Conclusion&Therefore the statement is false.\\
&\\
\hline
&\\
2.&\\
&\parbox{6cm}{\begin{align*}
    \mbox{Given that, }p(x)&=(x-1)^3\\
    \mbox{and }m(x)&=(x-1)^3\\
    \therefore \vec{J}&=\myvec{1&1&0\\0&1&1\\0&0&1}
\end{align*}}\\
&Since $\vec{J}$ is not diagonizable therefore $\vec{A}$ is not diagonizable.\\
&\\
\hline
&\\
Conclusion&Therefore the statement is false.\\
&\\
\hline
&\\
3.&\\
&Given that, $p(x)=(x-1)^3$\\
&Hence the eigen values of $\vec{A}$=1,1,1\\
&Hence the eigen values of $\vec{A}^2=1^2,1^2,1^2$ or 1,1,1\\
&Therefore the characteristic polynomial of $\vec{A}^2 = (x-1)^3$\\
&\\
\hline
&\\
Conclusion&Therefore the statement is True.\\
&\\
\hline
\pagebreak
\hline
&\\
4.&\\
&We know that jordan form of a matrix is similar to the original matrix\\
&Let $\vec{J}$ be the jordan form of the matrix $\vec{A}$ then,\\
&\parbox{6cm}{\begin{align*}
    \vec{A}&=\vec{P}\vec{J}\vec{P}^{-1}\\
    \vec{A}-\vec{I}&=\vec{P}\vec{J}\vec{P}^{-1}-\vec{I}\\
    \vec{A}-\vec{I}&=\vec{P}(\vec{J}-\vec{I})\vec{P}^{-1}\\
    (\vec{A}-\vec{I})^2&=\vec{P}(\vec{J}-\vec{I})\vec{P}^{-1}\vec{P}(\vec{J}-\vec{I})\vec{P}^{-1}\\
    (\vec{A}-\vec{I})^2&=\vec{P}(\vec{J}-\vec{I})^2\vec{P}^{-1}
\end{align*}}\\
&Therefore $(\vec{A}-\vec{I})^2$ is similar to $(\vec{J}-\vec{I})^2$\\
&Since $\vec{A}$ has exactly two jordan blocks and order of $\vec{A}$ is 3.\\
&\parbox{6cm}{\begin{align*}
    \therefore \vec{J}&=\myvec{1&1&0\\0&1&0\\0&0&1}\\
    \vec{J}-\vec{I}&=\myvec{0&1&0\\0&0&0\\0&0&0}\\
    (\vec{J}-\vec{I})^2&=\myvec{0&0&0\\0&0&0\\0&0&0}
\end{align*}}\\
&Since $(\vec{J}-\vec{I})^2$ is diagonal matrix.\\
&Therefore $(\vec{A}-\vec{I})^2$ is diagonalizable.\\
&\\
\hline
&\\
Conclusion&Therefore the statement is True.\\
&\\
\hline
\caption{Solution summary}
\label{table:2}
\end{longtable}
\end{document}
