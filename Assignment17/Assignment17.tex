
\documentclass[journal,12pt]{IEEEtran}
\usepackage{longtable}
\usepackage{setspace}
\usepackage{gensymb}
\singlespacing
\usepackage[cmex10]{amsmath}
\newcommand\myemptypage{
	\null
	\thispagestyle{empty}
	\addtocounter{page}{-1}
	\newpage
}
\usepackage{amsthm}
\usepackage{mdframed}
\usepackage{mathrsfs}
\usepackage{txfonts}
\usepackage{stfloats}
\usepackage{bm}
\usepackage{cite}
\usepackage{cases}
\usepackage{subfig}

\usepackage{longtable}
\usepackage{multirow}


\usepackage{enumitem}
\usepackage{mathtools}
\usepackage{steinmetz}
\usepackage{tikz}
\usepackage{circuitikz}
\usepackage{verbatim}
\usepackage{tfrupee}
\usepackage[breaklinks=true]{hyperref}
\usepackage{graphicx}
\usepackage{tkz-euclide}

\usetikzlibrary{calc,math}
\usepackage{listings}
    \usepackage{color}                                            %%
    \usepackage{array}                                            %%
    \usepackage{longtable}                                        %%
    \usepackage{calc}                                             %%
    \usepackage{multirow}                                         %%
    \usepackage{hhline}                                           %%
    \usepackage{ifthen}                                           %%
    \usepackage{lscape}     
\usepackage{multicol}
\usepackage{chngcntr}

\DeclareMathOperator*{\Res}{Res}

\renewcommand\thesection{\arabic{section}}
\renewcommand\thesubsection{\thesection.\arabic{subsection}}
\renewcommand\thesubsubsection{\thesubsection.\arabic{subsubsection}}

\renewcommand\thesectiondis{\arabic{section}}
\renewcommand\thesubsectiondis{\thesectiondis.\arabic{subsection}}
\renewcommand\thesubsubsectiondis{\thesubsectiondis.\arabic{subsubsection}}


\hyphenation{op-tical net-works semi-conduc-tor}
\def\inputGnumericTable{}                                 %%

\lstset{
%language=C,
frame=single, 
breaklines=true,
columns=fullflexible
}
\begin{document}
\onecolumn

\newtheorem{theorem}{Theorem}[section]
\newtheorem{problem}{Problem}
\newtheorem{proposition}{Proposition}[section]
\newtheorem{lemma}{Lemma}[section]
\newtheorem{corollary}[theorem]{Corollary}
\newtheorem{example}{Example}[section]
\newtheorem{definition}[problem]{Definition}

\newcommand{\BEQA}{\begin{eqnarray}}
\newcommand{\EEQA}{\end{eqnarray}}
\newcommand{\define}{\stackrel{\triangle}{=}}
\bibliographystyle{IEEEtran}
\raggedbottom
%\setlength{\parindent}{0pt}
\setlength\parskip{1em plus 2pt}
\providecommand{\mbf}{\mathbf}
\providecommand{\pr}[1]{\ensuremath{\Pr\left(#1\right)}}
\providecommand{\qfunc}[1]{\ensuremath{Q\left(#1\right)}}
\providecommand{\sbrak}[1]{\ensuremath{{}\left[#1\right]}}
\providecommand{\lsbrak}[1]{\ensuremath{{}\left[#1\right.}}
\providecommand{\rsbrak}[1]{\ensuremath{{}\left.#1\right]}}
\providecommand{\brak}[1]{\ensuremath{\left(#1\right)}}
\providecommand{\lbrak}[1]{\ensuremath{\left(#1\right.}}
\providecommand{\rbrak}[1]{\ensuremath{\left.#1\right)}}
\providecommand{\cbrak}[1]{\ensuremath{\left\{#1\right\}}}
\providecommand{\lcbrak}[1]{\ensuremath{\left\{#1\right.}}
\providecommand{\rcbrak}[1]{\ensuremath{\left.#1\right\}}}
\theoremstyle{remark}
\newtheorem{rem}{Remark}
\newcommand{\sgn}{\mathop{\mathrm{sgn}}}
\providecommand{\abs}[1]{\left\vert#1\right\vert}
\providecommand{\res}[1]{\Res\displaylimits_{#1}} 
\providecommand{\norm}[1]{\left\lVert#1\right\rVert}
%\providecommand{\norm}[1]{\lVert#1\rVert}
\providecommand{\mtx}[1]{\mathbf{#1}}
\providecommand{\mean}[1]{E\left[ #1 \right]}
\providecommand{\fourier}{\overset{\mathcal{F}}{ \rightleftharpoons}}
%\providecommand{\hilbert}{\overset{\mathcal{H}}{ \rightleftharpoons}}
\providecommand{\system}{\overset{\mathcal{H}}{ \longleftrightarrow}}
	%\newcommand{\solution}[2]{\textbf{Solution:}{#1}}
\newcommand{\solution}{\noindent \textbf{Solution: }}
\newcommand{\cosec}{\,\text{cosec}\,}
\providecommand{\dec}[2]{\ensuremath{\overset{#1}{\underset{#2}{\gtrless}}}}
\newcommand{\myvec}[1]{\ensuremath{\begin{pmatrix}#1\end{pmatrix}}}
\newcommand{\mydet}[1]{\ensuremath{\begin{vmatrix}#1\end{vmatrix}}}
\numberwithin{equation}{subsection}
\makeatletter
\@addtoreset{figure}{problem}
\makeatother
\let\StandardTheFigure\thefigure
\let\vec\mathbf
\renewcommand{\thefigure}{\theproblem}
\def\putbox#1#2#3{\makebox[0in][l]{\makebox[#1][l]{}\raisebox{\baselineskip}[0in][0in]{\raisebox{#2}[0in][0in]{#3}}}}
     \def\rightbox#1{\makebox[0in][r]{#1}}
     \def\centbox#1{\makebox[0in]{#1}}
     \def\topbox#1{\raisebox{-\baselineskip}[0in][0in]{#1}}
     \def\midbox#1{\raisebox{-0.5\baselineskip}[0in][0in]{#1}}
\vspace{3cm}
\title{Assignment 17}
\author{Utkarsh Surwade\\AI20MTECH11004}
\maketitle
\bigskip
\renewcommand{\thefigure}{\theenumi}
\renewcommand{\thetable}{\theenumi}
Download latex-tikz codes from 
%
\begin{lstlisting}
https://github.com/utkarshsurwade/Matrix_Theory_EE5609/tree/master/codes
\end{lstlisting}
%
 
\section{\textbf{Problem}}
Let $\vec{u}$ be a real n $\times$ 1 vector satisfying $\vec{u}^T\vec{u}=1$, where $\vec{u}^T$ is the transpose of $\vec{u}$.Define\\
$\vec{A}=\vec{I}-2\vec{u}\vec{u}^T$ where $\vec{I}$ is the $n^{th}$ order identity matrix.Which of the following statements are true?\\
1. $\vec{A}$ is singular\\
2. $\vec{A}^2=\vec{A}$\\
3. Trace($\vec{A}$)=n-2\\
4. $\vec{A}^2=\vec{I}$\\

\section{\textbf{Theorem 1.}}
Let $\vec{A}_{m \times n}$ and $\vec{B}_{n \times k}$ be matrices such that the product $\vec{AB}$ is well defines. Then\\
%Then rank($\vec{AB}$)$\leq$min( rank($\vec{A}$),rank($\vec{B}$) )\\
\begin{align}
    \mbox{rank}(\vec{AB})&\leq\mbox{min(rank}(\vec{A}),\mbox{rank}(\vec{B}))
\end{align}
Proof: Matrix $\vec{A}$ can be treated  as a linear transformation from $\mathbb{F}^n$ to $\mathbb{F}^m$.In that case rank of the matrix is the dimension of the image space of the transformation. If $\vec{T}$ is a linear
transformation from $\vec{V}_1$ to $\vec{V}_2$ then clearly dim $\vec{T}(\vec{V}_1)\leq$ dim ($\vec{V}_1$).Hence rank($\vec{AB}$) $\leq$ rank($\vec{B}$). Since row rank and column rank of a matrix are equal,
\begin{align}
    \mbox{Therefore rank}(\vec{AB})&\leq\mbox{min(rank}(\vec{A}),\mbox{rank}(\vec{B}))\label{eq:rank_of_AB}
\end{align}

\pagebreak
\section{\textbf{Definitions}}
\renewcommand{\thetable}{1}
\begin{table}[ht!]
\centering
\begin{tabular}{|c|l|}
    \hline
	\multirow{3}{*}{Characteristic Polynomial} 
	& \\
	& For an $n\times n$ matrix $\vec{A}$, characteristic polynomial is defined by,\\
	&\\
	& $\qquad\qquad\qquad p\brak{x}=\mydet{x\Vec{I}-\Vec{A}}$\\
	&\\
	\hline
	\multirow{3}{*}{Cayley-Hamilton Theorem}
    &\\
    & If $p\brak{x}$ is the characteristic polynomial of an $n\times n$ matrix $\vec{A}$, then,\\
    &\\
    &$\qquad \qquad \qquad p\brak{\vec{A}}=\vec{0}$\\
    &\\
    \hline
	\multirow{3}{*}{Minimal Polynomial} 
	&\\
	& Minimal polynomial $m\brak{x}$ is the smallest factor of\\
	&characteristic polynomial $p\brak{x}$ such that,\\
	&\\
	& $\qquad \qquad \qquad m\brak{\vec{A}}=0$\\
	& \\
	& Every root of characteristic polynomial should be the root of\\
	&minimal polynomial\\
	&\\
    \hline
\end{tabular}
\label{table:1}
    \caption{Definitions}
\end{table}
\newpage

\section{\textbf{Explanation}}
\renewcommand{\thetable}{2}
\begin{longtable}{|l|l|}
\hline
\multirow{3}{*}{} & \\
Statement&Solution\\
\hline
&\\
1.&\\
&\parbox{10cm}{\begin{align*}
    \mbox{Let }\vec{u}&=\myvec{u_1\\u_2\\\vdots\\u_n}\\
    \mbox{Let }\vec{B}&=\vec{u}\vec{u}^T\\
    \therefore \vec{B}&=\myvec{u_1\\u_2\\\vdots\\u_n}\myvec{u_1&u_2&\dots&u_n}\\
    \therefore \vec{B}&=\myvec{u_1^2&u_1u_2&\dots&u_1u_n\\
    u_2u_1&u_2^2&\dots&u_2u_n\\
    \vdots&\vdots&\ddots&\vdots\\
    u_nu_1&u_nu_2&\dots&u_n^2}\\
    \mbox{given that, }\vec{u}^T\vec{u}&=1\\
    \therefore \vec{u}^T\vec{u}&=\myvec{u_1&u_2&\dots&u_n}\myvec{u_1\\u_2\\\vdots\\u_n}\\
    \therefore \vec{u}^T\vec{u}&=u_1^2+u_2^2+\dots+u_n^2
\end{align*}}\\
&Since $\vec{u}$ is non-zero vector and $\vec{B}=\vec{u}\vec{u}^T$.\\
&Hence $\vec{B}$ is a non-zero matrix.\\
&Therefore Rank of $\vec{B}$ is at least 1.\\
&From \eqref{eq:rank_of_AB}\\
&\parbox{8cm}{\begin{align*}
    \mbox{rank}(\vec{B})&\leq\mbox{min(rank}(\vec{u}),\mbox{rank}(\vec{u}^T))\\
    \therefore\mbox{rank}(\vec{B})&\leq\mbox{min}(1,1)
\end{align*}}\\
&So Rank of $\vec{B}$ is at most 1.\\
&Hence Rank of $\vec{B}$ is equal to 1.\\
&Therefore $\vec{B}$ has n-1 eigenvalues equal to 0.\\
&Since the trace of a matrix is equal to the sum of its eigen values.\\
&We know that trace of $\vec{B}=u_1^2+u_2^2+\dots+u_n^2=1$\\
&\parbox{10cm}{\begin{align*}
    \therefore\mbox{Trace of }\vec{B}&=\lambda_1+\lambda_2+\dots+\lambda_{n-1}+\lambda_n\\
    1&=0+0+\dots+\lambda_n\\
    \therefore \lambda_n&=1
\end{align*}}\\
&Therefore the eigen values of $\vec{B}$ are $\lambda_1=0,\lambda_2=0,\dots,\lambda_{n-1}=0,\lambda_n=1$\\
&Hence the characteristic polynomial for $\vec{B}=x^{n-1}(x-1)$\\
&Since $\vec{A}=\vec{I}-2\vec{u}\vec{u}^T$\\
&and we know the eigen values of $\vec{I}$ are $\lambda_1=1,\lambda_2=1,\dots,\lambda_{n-1}=1,\lambda_n=1$\\
&and we know the eigen values of $\vec{u}\vec{u^T}$ are $\lambda_1=0,\lambda_2=0,\dots,\lambda_{n-1}=0,\lambda_n=1$\\
&\parbox{15cm}{\begin{align}
 \therefore\mbox{ The eigen values of }\vec{A}&=\lambda_1=1,\lambda_2=1,\dots,\lambda_{n-1}=1,\lambda_n=-1\label{eq:eigen_values_of_A}   
\end{align}}\\
&Since $\vec{A}$ does not have 0 as an eigen value\\
&Therefore $\vec{A}$ is not singular.\\
&\\
\hline
&\\
Conclusion&Therefore the statement is false.\\
&\\
\hline
&\\
2.&\\
& For $\vec{A}^2=\vec{A}$ ,\\
&we know that $p(x)=x^2-x$\\
&$\therefore$ minimal polynomial of $\vec{A}$ must divide x(x-1)\\
&$\therefore$ possible eigenvalues of $\vec{A}$ are 0 or 1\\
&But from \eqref{eq:eigen_values_of_A}, we know that $\vec{A}$ has -1 as an eigen value\\
&Therefore $\vec{A}^2=\vec{A}$ is false.\\
&\\
\hline
&\\
Conclusion&Therefore the statement is false.\\
&\\
\hline
&\\
3.&\\
& From equation \eqref{eq:eigen_values_of_A} ,\\
&Trace of $\vec{A}=n-2$\\
&\\
\hline
&\\
Conclusion&Therefore the statement is true.\\
&\\
\hline
&\\
4.&\\
&\parbox{6cm}{\begin{align*}
    \mbox{Since }\vec{A}&=\vec{I}-2\vec{u}\vec{u}^T\\
    \vec{A}^2&=(\vec{I}-2\vec{u}\vec{u}^T)(\vec{I}-2\vec{u}\vec{u}^T)\\
    \therefore\vec{A}^2&=\vec{I}-2\vec{u}\vec{u}^T-2\vec{u}\vec{u}^T+4\vec{u}\vec{u}^T\vec{u}\vec{u}^T\\
    \mbox{Since }\vec{u}^T\vec{u}&=1\\
    \therefore\vec{A}^2&=\vec{I}-2\vec{u}\vec{u}^T-2\vec{u}\vec{u}^T+4\vec{u}\vec{u}^T\\
    \therefore \vec{A}^2&=\vec{I}
\end{align*}}\\
&\\
\hline
&\\
Conclusion&Therefore the statement is true.\\
&\\
\hline
\caption{Solution summary}
\label{table:2}
\end{longtable}
\end{document}

