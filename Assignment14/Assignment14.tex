
\documentclass[journal,12pt]{IEEEtran}
%\documentclass{proc}

\usepackage{longtable}
\usepackage{setspace}
\usepackage{gensymb}
\singlespacing
\usepackage[cmex10]{amsmath}
\newcommand\myemptypage{
	\null
	\thispagestyle{empty}
	\addtocounter{page}{-1}
	\newpage
}
\usepackage{amsthm}
\usepackage{mdframed}
\usepackage{mathrsfs}
\usepackage{txfonts}
\usepackage{stfloats}
\usepackage{bm}
\usepackage{cite}
\usepackage{cases}
\usepackage{subfig}

\usepackage{longtable}
\usepackage{multirow}


\usepackage{enumitem}
\usepackage{mathtools}
\usepackage{steinmetz}
\usepackage{tikz}
\usepackage{circuitikz}
\usepackage{verbatim}
\usepackage{tfrupee}
\usepackage[breaklinks=true]{hyperref}
\usepackage{graphicx}
\usepackage{tkz-euclide}

\usetikzlibrary{calc,math}
\usepackage{listings}
    \usepackage{color}                                            %%
    \usepackage{array}                                            %%
    \usepackage{longtable}                                        %%
    \usepackage{calc}                                             %%
    \usepackage{multirow}                                         %%
    \usepackage{hhline}                                           %%
    \usepackage{ifthen}                                           %%
    \usepackage{lscape}     
\usepackage{multicol}
\usepackage{chngcntr}

\DeclareMathOperator*{\Res}{Res}

\renewcommand\thesection{\arabic{section}}
\renewcommand\thesubsection{\thesection.\arabic{subsection}}
\renewcommand\thesubsubsection{\thesubsection.\arabic{subsubsection}}

\renewcommand\thesectiondis{\arabic{section}}
\renewcommand\thesubsectiondis{\thesectiondis.\arabic{subsection}}
\renewcommand\thesubsubsectiondis{\thesubsectiondis.\arabic{subsubsection}}


\hyphenation{op-tical net-works semi-conduc-tor}
\def\inputGnumericTable{}                                 %%

\lstset{
%language=C,
frame=single, 
breaklines=true,
columns=fullflexible
}
\begin{document}
\onecolumn

\newtheorem{theorem}{Theorem}[section]
\newtheorem{problem}{Problem}
\newtheorem{proposition}{Proposition}[section]
\newtheorem{lemma}{Lemma}[section]
\newtheorem{corollary}[theorem]{Corollary}
\newtheorem{example}{Example}[section]
\newtheorem{definition}[problem]{Definition}

\newcommand{\BEQA}{\begin{eqnarray}}
\newcommand{\EEQA}{\end{eqnarray}}
\newcommand{\define}{\stackrel{\triangle}{=}}
\bibliographystyle{IEEEtran}
\raggedbottom
\setlength{\parindent}{0pt}
\providecommand{\mbf}{\mathbf}
\providecommand{\pr}[1]{\ensuremath{\Pr\left(#1\right)}}
\providecommand{\qfunc}[1]{\ensuremath{Q\left(#1\right)}}
\providecommand{\sbrak}[1]{\ensuremath{{}\left[#1\right]}}
\providecommand{\lsbrak}[1]{\ensuremath{{}\left[#1\right.}}
\providecommand{\rsbrak}[1]{\ensuremath{{}\left.#1\right]}}
\providecommand{\brak}[1]{\ensuremath{\left(#1\right)}}
\providecommand{\lbrak}[1]{\ensuremath{\left(#1\right.}}
\providecommand{\rbrak}[1]{\ensuremath{\left.#1\right)}}
\providecommand{\cbrak}[1]{\ensuremath{\left\{#1\right\}}}
\providecommand{\lcbrak}[1]{\ensuremath{\left\{#1\right.}}
\providecommand{\rcbrak}[1]{\ensuremath{\left.#1\right\}}}
\theoremstyle{remark}
\newtheorem{rem}{Remark}
\newcommand{\sgn}{\mathop{\mathrm{sgn}}}
\providecommand{\abs}[1]{\left\vert#1\right\vert}
\providecommand{\res}[1]{\Res\displaylimits_{#1}} 
\providecommand{\norm}[1]{\left\lVert#1\right\rVert}
%\providecommand{\norm}[1]{\lVert#1\rVert}
\providecommand{\mtx}[1]{\mathbf{#1}}
\providecommand{\mean}[1]{E\left[ #1 \right]}
\providecommand{\fourier}{\overset{\mathcal{F}}{ \rightleftharpoons}}
%\providecommand{\hilbert}{\overset{\mathcal{H}}{ \rightleftharpoons}}
\providecommand{\system}{\overset{\mathcal{H}}{ \longleftrightarrow}}
	%\newcommand{\solution}[2]{\textbf{Solution:}{#1}}
\newcommand{\solution}{\noindent \textbf{Solution: }}
\newcommand{\cosec}{\,\text{cosec}\,}
\providecommand{\dec}[2]{\ensuremath{\overset{#1}{\underset{#2}{\gtrless}}}}
\newcommand{\myvec}[1]{\ensuremath{\begin{pmatrix}#1\end{pmatrix}}}
\newcommand{\mydet}[1]{\ensuremath{\begin{vmatrix}#1\end{vmatrix}}}
\numberwithin{equation}{subsection}
\makeatletter
\@addtoreset{figure}{problem}
\makeatother
\let\StandardTheFigure\thefigure
\let\vec\mathbf
\renewcommand{\thefigure}{\theproblem}
\def\putbox#1#2#3{\makebox[0in][l]{\makebox[#1][l]{}\raisebox{\baselineskip}[0in][0in]{\raisebox{#2}[0in][0in]{#3}}}}
     \def\rightbox#1{\makebox[0in][r]{#1}}
     \def\centbox#1{\makebox[0in]{#1}}
     \def\topbox#1{\raisebox{-\baselineskip}[0in][0in]{#1}}
     \def\midbox#1{\raisebox{-0.5\baselineskip}[0in][0in]{#1}}
\vspace{3cm}
\title{Assignment 14}
\author{Utkarsh Surwade\\AI20MTECH11004}
\maketitle
\bigskip
\renewcommand{\thefigure}{\theenumi}
\renewcommand{\thetable}{\theenumi}
Download latex-tikz codes from 
%
\begin{lstlisting}
https://github.com/utkarshsurwade/Matrix_Theory_EE5609/tree/master/codes
\end{lstlisting}
%
 
\section{\textbf{Problem}}
Let $\vec{A}$ be an n $\times$ m matrix with each entry equal to +1,-1 or 0 such that every column has exactly one +1 and exactly one -1. We can conclude that\\
\begin{align}
    &\mbox{1. Rank } \vec{A}\leq n-1\\
    &\mbox{2. Rank } \vec{A}=m\\    
    &\mbox{3. }n\leq m\\
    &\mbox{4. }n-1\leq m
\end{align}
\section{\textbf{Explanation}}
\renewcommand{\thetable}{1}
\begin{longtable}{|l|l|}
\hline
\multirow{3}{*}{} & \\
option&Solution\\
\hline
&\\
1.&Let us consider $\vec{A}$ as follows and let s be the summation of all column entries:\\
&\parbox{6cm}{\begin{align*}
    \vec{A}&=\myvec{a_{11}&a_{12}&\dots&a_{1m}\\a_{21}&a_{22}&\dots&a_{2m}\\\vdots&\vdots&\vdots&\vdots\\a_{n1}&a_{n2}&\dots&a_{nm}}\\
    \mydet{\vec{A}-\lambda\vec{I}}&=\myvec{a_{11}-\lambda&a_{12}&\dots&a_{1m}\\a_{21}&a_{22}-\lambda&\dots&a_{2m}\\\vdots&\vdots&\vdots&\vdots\\a_{n1}&a_{n2}&\dots&a_{nm}-\lambda}=0\\
    &=\myvec{a_{11}+a_{21}+\dots+a{n1}-\lambda&a_{11}+a_{21}+\dots+a{n1}-\lambda&\dots&a_{11}+a_{21}+\dots+a{n1}-\lambda\\
    a_{21}&a_{22}-\lambda&\dots&a_{2m}\\\vdots&\vdots&\vdots&\vdots\\a_{n1}&a_{n2}&\dots&a_{nm}-\lambda}\\
    &\implies (s-\lambda)\myvec{1&1&\dots&1\\a_{21}&a_{22}-\lambda&\dots&a_{2m}\\\vdots&\vdots&\vdots&\vdots\\a_{n1}&a_{n2}&\dots&a_{nm}-\lambda}=0\\
\end{align*}}\\
\hline
\pagebreak
\hline
&\\
& Since s=0 according to question,\\
&Therefore $\lambda=0$ is an eigen value of $\vec{A}$.\\
&Since $\lambda=0$, Hence $\vec{A}$ is singular.\\
&Which means at least two rows are linearly dependent.\\
&Therefore,\\
&\parbox{6cm}{\begin{align*}
    \mbox{Rank}(\vec{A}) &< n\\
    \mbox{Rank}(\vec{A}) &\leq n-1
\end{align*}}\\
Example&Let us Consider $\vec{A}$ as follows,where n=4 and m=3\\
&\parbox{6cm}{\begin{align*}
    \vec{A}=\myvec{1&0&0\\0&1&0\\0&0&1\\-1&-1&-1}
\end{align*}}\\
&Calculating Row Reduced Echelon Form of $\vec{A}$ as follows:\\
&\parbox{6cm}{\begin{align*}
    \xleftrightarrow[R_4 \leftarrow R_2+R_4]{R_4 \leftarrow R_1+R_4}
		\myvec{1&0&0\\0&1&0\\0&0&1\\0&0&-1}\\
	\xleftrightarrow[]{R_4 \leftarrow R_3+R_4}
		\myvec{1&0&0\\0&1&0\\0&0&1\\0&0&0}
\end{align*}}\\
\hline
&\\
Conclusion&Since the Rank  $\vec{A}$=3 and n=4,\\
&Therefore the Rank $\vec{A} \leq n-1$ statement is true.\\
&\\
\hline
&\\
2.&Let us Consider $\vec{A}$ as follows,where n=2 and m=2\\
&\parbox{6cm}{\begin{align*}
    \vec{A}=\myvec{-1&1\\1&-1}
\end{align*}}\\
&Applying elementary transformations on $\vec{A}$ as follows:\\
&\parbox{6cm}{\begin{align*}
    \xleftrightarrow[]{R_2 \leftarrow R_1+R_2}
		\myvec{-1&1\\0&0}
\end{align*}}\\
\hline
&\\
Conclusion&Since the Rank  $\vec{A}$=1 and m=2,\\
&Therefore the Rank $\vec{A} \neq m$, Hence the statement is false.\\
&\\
\hline
\pagebreak
\hline
&\\
3.&Let us Consider $\vec{A}$ as follows,where n=3 and m=2\\
&\parbox{6cm}{\begin{align}
    \vec{A}=\myvec{1&1\\-1&-1\\0&0}
\end{align}}\\
\hline
&\\
Conclusion&Since there exists a matrix $\vec{A}$ when n$>$m,\\
&Therefore the statement is false.\\
&\\
\hline
&\\
4&Let us Consider $\vec{A}$ as follows,where n=4 and m=2\\
&\parbox{6cm}{\begin{align}
    \vec{A}=\myvec{1&1\\-1&-1\\0&0\\0&0}
\end{align}}\\
\hline
&\\
Conclusion&Since there exists a matrix $\vec{A}$ when n-1$>$m,\\
&Therefore the statement is false.\\
&\\
\hline
\caption{Solution summary}
\label{table:2}
\end{longtable}
\end{document}

